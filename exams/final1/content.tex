\newcommand\UniversiteAdi{Niğde Ömer Halisdemir Üniversitesi}
\newcommand\BolumAdi{MEKATRONİK BÖLÜMÜ}
\newcommand\DersKodu{MKT2002}
\newcommand\DersAdi{BİLGİSAYARLI KONTROL SİSTEMLERİ}
\newcommand\SinavAdi{Genel Sınav}
\newcommand\SinavTarihi{10.03.2025}
\newcommand\SinavSaati{10:00}
\newcommand\SinavSuresi{90dk}

\pagestyle{fancy}
\fancyhf{} % clear existing header/footer entries
\fancyhead[R]{Öğrenci No:\hspace{4.5cm}}
\fancyhead[L]{Ad Soyad:\hspace{7cm}}
\noindent
\begin{tabular}{
    p{0.15\linewidth}
    p{0.15\linewidth}
    p{0.3\linewidth}
    p{0.1\linewidth}
    p{0.15\linewidth}}
    \multicolumn{5}{c}{\textbf{\BolumAdi}}\\
    \multicolumn{5}{c}{\textbf{\DersAdi}}\\\hline
    \multicolumn{1}{|r|}{Ders Kodu:}&
    \multicolumn{1}{|c|}{\DersKodu}&
    \multicolumn{1}{|c|}{}& 
    \multicolumn{1}{|r|}{Tarih:}&
    \multicolumn{1}{|c|}{\SinavTarihi} \\\hline
    \multicolumn{1}{|r|}{Sınav Türü:}&
    \multicolumn{1}{|c|}{\SinavAdi}&  
    \multicolumn{1}{|c|}{}&
    \multicolumn{1}{|r|}{Saat:}&
    \multicolumn{1}{|c|}{\SinavSaati}\\\hline
    \multicolumn{1}{|r|}{Dönemi:}&
    \multicolumn{1}{|c|}{2024-2025}&
    \multicolumn{1}{|c|}{}&
    \multicolumn{1}{|r|}{Süre:}&
    \multicolumn{1}{|c|}{\SinavSuresi} \\\hline
    &&&&\\
\end{tabular}\\\\
\noindent\begin{center}
\begin{tabular}{|r|c|c|c|c|}\hline
    \textbf{Soru:}&
    \textbf{1}&
    \textbf{2}&
    \textbf{3}&
    \textbf{Toplam}\\\hline
    \textbf{Puan:}&
    \textbf{35}&
    \textbf{35}&
    \textbf{30}&
    \textbf{100}\\\hline
    \textbf{Not:}&&&&\\\hline
\end{tabular}\end{center}
\noindent\textbf{Uyarı:}
\begin{itemize}\bfseries
    \item Soruları dikkatlice okuyunuz. Hesap makinesi kullanılabilir.
    \item Defter, kitap ve notlar açık bir sınavdır.
    \item İşlemleri atlamadan ve ayrıntılı olarak veriniz. Sadece nümerik yanıtlar veya çizimler ara işlemler olmadan kabul edilmemektedir.
    \item Yuvarlamalar 2 hane yapılacaktır. $\mathbf{1.99456\approx1.99}$ olarak alınacaktır.
\end{itemize}

\begin{enumerate}[\bfseries S1.]
    \item (35p)\,Birinci dereceden bir sistem 
    \begin{equation}
        G(z)=\frac{1}{z+1.2}
    \end{equation}
    olarak verilmiştir. $t_s=4\,s$ ve aşım $\%16.3$ olacak şekilde bir ayrık PI kontrolör tasarlayınız.
    \item (35p)\,Ayrık bir durum uzayı
    \begin{equation}
        A=\begin{bmatrix}1& 1\\-1&0\end{bmatrix},\,B=\begin{bmatrix}0\\1\end{bmatrix},\,C=\begin{bmatrix}1&0\end{bmatrix}
    \end{equation}
    olarak verilmiştir. Durum geri besleme kontrolörü için amaçlanan kapalı çevrim karakteristikleri
    \begin{equation}
        p(z)=z^2+0.5z+0.25
    \end{equation}
    ile ifade edilmektedir. Durum geri beleme kontrolörünü elde ediniz.
    \begin{equation*}
        \begin{bmatrix}0& 1\\1&-1\end{bmatrix}^{-1}=\begin{bmatrix}1& 1\\1&0\end{bmatrix},\,
        \begin{bmatrix}1& 1\\-1&0\end{bmatrix}^{-1}=\begin{bmatrix}-1& -1\\1&0\end{bmatrix},\,
        \begin{bmatrix}1& 1\\-1&0\end{bmatrix}\begin{bmatrix}1& 1\\-1&0\end{bmatrix}=\begin{bmatrix}-1& -1\\1&0\end{bmatrix}
    \end{equation*}
    \item (30p)\,S tanım bölgesinde verilen 
    \begin{equation}
        G(s)=\frac{1}{(s+1)(s+2)(s+3)}
    \end{equation}
    ifadeyi z tanım bölgesine dönüştürünüz. 
    \begin{equation*}
    \begin{split}
        \begin{bmatrix}
            1&     1&     1\\
            4&     5&     3\\
            3&     6&     2\\
        \end{bmatrix}^{-1}=\begin{bmatrix}
            -4 &   2   &    -1\\
            0.5&   -0.5&    0.5\\
            4.5&   -1.5&    0.5\\
        \end{bmatrix},\,
        \begin{bmatrix}
            1&     1&     1\\
            4&     3&     5\\
            3&     2&     6\\
        \end{bmatrix}^{-1}=\begin{bmatrix}
            -4&    2&   -1\\
            4.5&   -1.5&    0.5\\
            0.5&  -0.5&    0.5\\
        \end{bmatrix}\\
        \begin{bmatrix}
            1&     1&     1\\
            5&     4&     3\\
            6&     3&     2\\
        \end{bmatrix}^{-1}=\begin{bmatrix}
        0.5 &  -0.5 &   0.5\\
        -4&    2&   -1.\\
        4.5 &  -1.5 &   0.5\\
        \end{bmatrix},\,
        \begin{bmatrix}
            1&     1&     1\\
            5&     3&     4\\
            6&     2&     3\\
        \end{bmatrix}^{-1}=\begin{bmatrix}
            0.5&   -0.5&    0.5\\
            4.5&   -1.5&    0.5\\
             -4&    2  &    -1\\
        \end{bmatrix}\\
        \begin{bmatrix}
            1&     1&     1\\
            3&     4&     5\\
            2&     3&     6\\
        \end{bmatrix}^{-1}=\begin{bmatrix}
            4.5 &  -1.5&    0.5\\
            -4  &    2 &     -1\\
             0.5&  -0.5&    0.5\\
        \end{bmatrix},\,
        \begin{bmatrix}
            1&     1&     1\\
            3&     5&     4\\
            2&     6&     3\\
        \end{bmatrix}^{-1}=\begin{bmatrix}
            4.5&   -1.5&    0.5\\
            0.5&   -0.5&    0.5\\
           -4&    2&   -1\\
        \end{bmatrix}\\
    \end{split}
    \end{equation*}
\end{enumerate}
