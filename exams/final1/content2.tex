\newcommand\UniversiteAdi{Niğde Ömer Halisdemir Üniversitesi}
\newcommand\BolumAdi{MEKATRONİK BÖLÜMÜ}
\newcommand\DersKodu{MKT2002}
\newcommand\DersAdi{BİLGİSAYARLI KONTROL SİSTEMLERİ}
\newcommand\SinavAdi{Genel Sınav}
\newcommand\SinavTarihi{10.05.2025}
\newcommand\SinavSaati{10:00}
\newcommand\SinavSuresi{90dk}

\pagestyle{fancy}
\fancyhf{} % clear existing header/footer entries
\fancyhead[R]{Öğrenci No:\hspace{4.5cm}}
\fancyhead[L]{Ad Soyad:\hspace{7cm}}
\noindent
\begin{tabular}{
    p{0.15\linewidth}
    p{0.15\linewidth}
    p{0.3\linewidth}
    p{0.1\linewidth}
    p{0.15\linewidth}}
    \multicolumn{5}{c}{\textbf{\BolumAdi}}\\
    \multicolumn{5}{c}{\textbf{\DersAdi}}\\\hline
    \multicolumn{1}{|r|}{Ders Kodu:}&
    \multicolumn{1}{|c|}{\DersKodu}&
    \multicolumn{1}{|c|}{}& 
    \multicolumn{1}{|r|}{Tarih:}&
    \multicolumn{1}{|c|}{\SinavTarihi} \\\hline
    \multicolumn{1}{|r|}{Sınav Türü:}&
    \multicolumn{1}{|c|}{\SinavAdi}&  
    \multicolumn{1}{|c|}{}&
    \multicolumn{1}{|r|}{Saat:}&
    \multicolumn{1}{|c|}{\SinavSaati}\\\hline
    \multicolumn{1}{|r|}{Dönemi:}&
    \multicolumn{1}{|c|}{2024-2025}&
    \multicolumn{1}{|c|}{}&
    \multicolumn{1}{|r|}{Süre:}&
    \multicolumn{1}{|c|}{\SinavSuresi} \\\hline
    &&&&\\
\end{tabular}\\\\
\noindent\begin{center}
\begin{tabular}{|r|c|c|c|c|}\hline
    \textbf{Soru:}&
    \textbf{1}&
    \textbf{2}&
    \textbf{3}&
    \textbf{Toplam}\\\hline
    \textbf{Puan:}&
    \textbf{35}&
    \textbf{35}&
    \textbf{30}&
    \textbf{100}\\\hline
    \textbf{Not:}&35&35&30&100\\\hline
\end{tabular}\end{center}
\noindent\textbf{Uyarı:}
\begin{itemize}\bfseries
    \item Soruları dikkatlice okuyunuz. Hesap makinesi kullanılabilir.
    \item Defter, kitap ve notlar açık bir sınavdır.
    \item İşlemleri atlamadan ve ayrıntılı olarak veriniz. Sadece nümerik yanıtlar veya çizimler ara işlemler olmadan kabul edilmemektedir.
    \item Yuvarlamalar 2 hane yapılacaktır. $\mathbf{1.99456\approx1.99}$ olarak alınacaktır.
\end{itemize}

\begin{enumerate}[\bfseries S1.]
    \item (35p)\,Örnekleme zamanı $T=1\,s$ olan birinci dereceden bir sistem 
    \begin{equation}
        G(z)=\frac{1}{z+1.2}
    \end{equation}
    olarak verilmiştir. $t_s=4\,s$ ve aşım $\%16.3$ olacak şekilde bir ayrık PI kontrolör tasarlayınız. Aşım isterinden hareketle
    \begin{equation}
    \begin{split}
        \zeta&=-\frac{log_e(0.163)}{\sqrt{\pi^2+log_e(0.163)^2}}\\
        &=0.5
    \end{split}
    \end{equation}
    ve yerleşme zamanından hareketle ise 
    \begin{equation}
        \begin{split}
            w_n&=\frac{4}{t_s\zeta}\\
            w_n&=\frac{4}{4\cdot 0.5}\\
            w_n&=2
        \end{split}
    \end{equation}
    elde edilir. S tanım bölgesinde kutuplar
    \begin{equation}
    \begin{split}
        s&=-\zeta w_n+\sqrt{1-\zeta^2}w_n\\
        s&=-1\pm \sqrt{3}i\\
        s&=-1\pm 1.73i
    \end{split}
    \end{equation}
    olarak hesaplanır. Z tanım bölgesine geçilince
    \begin{equation}
        \begin{split}
            z&=e^{s}\\
            &=e^{-1\pm 1.73i}\\
            &=e^{-1}\phase{1.73}\\
            &=e^{-1}\phase{99.12^o}\\
            &=0.39\pm0.46i
        \end{split}
    \end{equation}
    elde edilir. Kapalı çevrim için aday polinom
    \begin{equation}
        p(z)=z^2-0.79z+0.37
    \end{equation}
    şeklindedir. PI kontrolör
    \begin{equation}
        F(z)=\frac{k_iz+k_p}{z-1}
    \end{equation}
    olmak üzere kapalı çevrim transfer fonksiyonu
    \begin{equation}
    \begin{split}
        T(z)&=\frac{\frac{k_iz+k_p}{z-1}\frac{1}{z+1.2}}{1+\frac{k_iz+k_p}{z-1}\frac{1}{z+1.2}}\\
        &=\frac{k_iz+k_p}{z^2+0.2z-1.2+k_iz+k_p}\\
        &=\frac{k_iz+k_p}{z^2+(0.2+k_i)z+k_p-1.2}
    \end{split}
    \end{equation}
    şeklindedir. Bu durumda tasarım problemi
    \begin{equation}
        \begin{split}
        -0.79 &= 0.2 + k_i\\
         0.37 &= -1.2 + k_p
        \end{split}
    \end{equation}
    ve çözüm $k_p=-0.99$ ve $k_p=1.57$ olarak elde edilir. Sonuç olarak PI kontrolör
    \begin{equation}
        F(z)=\frac{-0.99z+1.57}{z-1}=\frac{-0.99(z-1.59)}{z-1}
    \end{equation}
    olarak hesaplanır.

    \item (35p)\,Ayrık bir durum uzayı
    \begin{equation}
        A=\begin{bmatrix}1& 1\\-1&0\end{bmatrix},\,B=\begin{bmatrix}0\\1\end{bmatrix},\,C=\begin{bmatrix}1&0\end{bmatrix}
    \end{equation}
    olarak verilmiştir. Durum geri besleme kontrolörü için amaçlanan kapalı çevrim karakteristikleri
    \begin{equation}
        p(z)=z^2+0.5z+0.25
    \end{equation}
    ile ifade edilmektedir. Durum geri beleme kontrolörünü elde ediniz.
    \begin{equation*}
        \begin{bmatrix}0& 1\\1&-1\end{bmatrix}^{-1}=\begin{bmatrix}1& 1\\1&0\end{bmatrix},\,
        \begin{bmatrix}1& 1\\-1&0\end{bmatrix}^{-1}=\begin{bmatrix}-1& -1\\1&0\end{bmatrix},\,
        \begin{bmatrix}1& 1\\-1&0\end{bmatrix}\begin{bmatrix}1& 1\\-1&0\end{bmatrix}=\begin{bmatrix}-1& -1\\1&0\end{bmatrix}
    \end{equation*}
    $p_d(A)$ terimi
    \begin{equation}
    \begin{split}
        p_d(A)&=A\cdot A+0.5A+0.25I\\
        &=\begin{bmatrix}
            -1&    -1\\
            1&    0\\
        \end{bmatrix}+\begin{bmatrix}
        0& 0.5\\-0.5& -0.5\end{bmatrix}
        +\begin{bmatrix}
            0.25& 0\\0.25& 0\end{bmatrix}\\
        &=\begin{bmatrix}
            -0.75&   -0.5\\
            0.5&   -0.25
        \end{bmatrix}
    \end{split}
    \end{equation}
    ve $\Phi$ ise 
    \begin{equation}
        \begin{split}
            \Phi&=[B\, AB]\\
            \Phi&=\begin{bmatrix}
            0& 1\\
            1& -1
            \end{bmatrix}
        \end{split}
    \end{equation}
    şeklinde hesaplanmaktadır. Tersi ise
    \begin{equation}
        \begin{split}
            \Phi^{-1}&=\begin{bmatrix}
            0& 1\\
            1& -1
            \end{bmatrix}^{-1}\\
            \Phi^{-1}&=\begin{bmatrix}
                1&     1\\
                1&     0
            \end{bmatrix}
        \end{split}
    \end{equation}
    olarak soruda verilmiştir. Ackermann formülü gereğince durum geri besleme kontrolörü
    \begin{equation}
        \begin{split}
            K&=-\begin{bmatrix}0& 1\end{bmatrix}\Phi^{-1}p_d(A)\\
            &=-\begin{bmatrix}0& 1\end{bmatrix}\begin{bmatrix}
                1&     1\\
                1&     0
            \end{bmatrix}\begin{bmatrix}
                -0.75&   -0.5\\
                0.5&   -0.25
            \end{bmatrix}\\
            &=\begin{bmatrix}0.75& 0.5\end{bmatrix}
        \end{split}
    \end{equation}
    şeklinde elde edilir.

    \item (30p)\,S tanım bölgesinde verilen 
    \begin{equation}
        G(s)=\frac{1}{(s+1)(s+2)(s+3)}=\frac{A}{s+2}+\frac{B}{s+1}+\frac{C}{s+3}
    \end{equation}
    ifadeyi z tanım bölgesine dönüştürünüz. 
    \begin{equation*}
        \begin{split}
            \begin{bmatrix}
                1&     1&     1\\
                4&     5&     3\\
                3&     6&     2\\
            \end{bmatrix}^{-1}=\begin{bmatrix}
                -4 &   2   &    -1\\
                0.5&   -0.5&    0.5\\
                4.5&   -1.5&    0.5\\
            \end{bmatrix},\,
            \begin{bmatrix}
                1&     1&     1\\
                4&     3&     5\\
                3&     2&     6\\
            \end{bmatrix}^{-1}=\begin{bmatrix}
                -4&    2&   -1\\
                4.5&   -1.5&    0.5\\
                0.5&  -0.5&    0.5\\
            \end{bmatrix}\\
            \begin{bmatrix}
                1&     1&     1\\
                5&     4&     3\\
                6&     3&     2\\
            \end{bmatrix}^{-1}=\begin{bmatrix}
            0.5 &  -0.5 &   0.5\\
            -4&    2&   -1.\\
            4.5 &  -1.5 &   0.5\\
            \end{bmatrix},\,
            \begin{bmatrix}
                1&     1&     1\\
                5&     3&     4\\
                6&     2&     3\\
            \end{bmatrix}^{-1}=\begin{bmatrix}
                0.5&   -0.5&    0.5\\
                4.5&   -1.5&    0.5\\
                 -4&    2  &    -1\\
            \end{bmatrix}\\
            \begin{bmatrix}
                1&     1&     1\\
                3&     4&     5\\
                2&     3&     6\\
            \end{bmatrix}^{-1}=\begin{bmatrix}
                4.5 &  -1.5&    0.5\\
                -4  &    2 &     -1\\
                 0.5&  -0.5&    0.5\\
            \end{bmatrix},\,
            \begin{bmatrix}
                1&     1&     1\\
                3&     5&     4\\
                2&     6&     3\\
            \end{bmatrix}^{-1}=\begin{bmatrix}
                4.5&   -1.5&    0.5\\
                0.5&   -0.5&    0.5\\
               -4&    2&   -1\\
            \end{bmatrix}\\
        \end{split}
        \end{equation*}
    Verilen ifadeden 
    \begin{equation}
    \begin{split}
        \frac{1}{(s+1)(s+2)(s+3)}&=\frac{A}{s+2}+\frac{B}{s+1}+\frac{C}{s+3}\\
        A(s+1)(s+3)+B(s+2)(s+3)+C(s+1)(s+2)&=1\\
        A(s^2+4s+3)+B(s^2+5s+6)+C(s^2+3s+2)&=1\\
        As^2+4As+3A+Bs^2+5Bs+6B+Cs^2+3Cs+2C&=1\\
        (A+B+C)s^2+(4A+5B+3C)s+(3A+6B+2C)&=1
    \end{split}
    \end{equation}
    ve dolayısıyla  
    \begin{equation}
        \begin{split}
            A+B+C&=0\\
            4A+5B+3C&=0\\
            3A+6B+2C&=1
        \end{split}
    \end{equation}
    elde edilir. Matrisin tersi kullanılarak
    \begin{equation}
        \begin{split}
        \begin{bmatrix}
            1& 1& 1\\
            4& 5& 3\\
            3& 6& 2
        \end{bmatrix}\begin{bmatrix}A\\B\\C\end{bmatrix}&=\begin{bmatrix}0\\0\\1\end{bmatrix}\\
        \begin{bmatrix}A\\B\\C\end{bmatrix}&=\begin{bmatrix}
            1& 1& 1\\
            4& 5& 3\\
            3& 6& 2
        \end{bmatrix}^{-1}\begin{bmatrix}0\\0\\1\end{bmatrix}\\
        \begin{bmatrix}A\\B\\C\end{bmatrix}&=\begin{bmatrix}
            -4 &   2   &    -1\\
            0.5&   -0.5&    0.5\\
            4.5&   -1.5&    0.5\\
        \end{bmatrix}\begin{bmatrix}0\\0\\1\end{bmatrix}\\
        \begin{bmatrix}A\\B\\C\end{bmatrix}&=\begin{bmatrix}
        -1\\
        0.5\\
        0.5\\
        \end{bmatrix}
        \end{split}
    \end{equation}
    Çözüm ile
    \begin{equation}
        \begin{split}
            \frac{1}{(s+1)(s+2)(s+3)}&=-\frac{1}{s+2}+\frac{1}{2}\cdot\frac{1}{s+1}+\frac{1}{2}\cdot\frac{1}{s+3}\\
        \end{split}
    \end{equation}
    ifadesi yazılır.

\end{enumerate}
