\chapter{Fark Denklemleri}
Örnek sistemin ZOH yöntemi ile elde edilen ve Denklem~\ref{eqn:ornek_sistem_zoh} ile verilen sistem için
\begin{equation}
\begin{split}
    G_{ZOH}(z)&=\frac{1-e^{-1}}{z-e^{-1}}\\
    &=\frac{(1-e^{-1})z^{-1}}{1-e^{-1}z^{-1}}\\
    \frac{y(z)}{u(z)}&=\frac{(1-e^{-1})z^{-1}}{1-e^{-1}z^{-1}}\\
    y(z)(1-e^{-1}z^{-1})&=\frac{(1-e^{-1})z^{-1}u(z)}{1-e^{-1}z^{-1}}\\
    y(z)-y(z-1)e^{-1}&=(1-e^{-1})u(z-1)\\
    y(z)&=y(z-1)e^{-1}+(1-e^{-1})u(z-1)\\
    y(z)&=0.3679y(z-1)+0.6321u(z-1)
\end{split}
\end{equation}
elde edilir. Z tanım bölgesinde tanımlı transfer fonksiyonundan fark denklemine geçişe örnektir. Fark denklemleri programlama dilleri ile kolaylıkla gerçeklenebilmektedir.
\begin{lstlisting}
u=ones(1,length(t));
y=zeros(1,length(t));

for i=2:length(t)
    y(i)=exp(-T)*y(i-1)+(1-exp(-T))*u(i);
end\end{lstlisting}
Benzer şekilde FOH yöntemi ile elde edilen ve Denklem~\ref{eqn:ornek_sistem_foh} ile verilen ifade için
\begin{equation}
    \begin{split}
        G_{FOH}(z)&=\frac{1}{z}\\
        \frac{y(z)}{u(z)}&=z^{-1}\\
        y(z)&=u(z-1)
    \end{split}
\end{equation}
elde edilir.
Yay-Kütle-Damper sistemi için dinamikleri ifade eden denklem
\begin{equation}
    m\ddot{x}(t)+b\dot{x}(t)+kx(t)=u(t)
\end{equation}
olarak verilmiştir. Bu diferansiyel denklem S tanım bölgesine dönüştürülürse
\begin{equation}
\begin{split}
    ms^2X(s)+b sX(s)+kX(s)&=U(s)\\
    (ms^2+b s+k)X(s)&=U(s)\\
    \frac{X(s)}{U(s)}=\frac{1}{ms^2+b s+k}
\end{split}
\end{equation}
elde edilir.