\chapter{Z Tanım Bölgesinde Durum Geri Besleme Kontrolörü}
Durum uzay modeli
\begin{equation}
    x(k)=A x(k-1)+Bu(k-1),\quad y(k-1)=C x(k-1)
\end{equation}
olmak üzere 
\begin{equation}
    u(k-1)=K x(k-1)
\end{equation}
kontrolörüne \textbf{Durum Geri Besleme} kontrolörü adı verilmektedir. Dikkat edilirse bu kontrol kuralı
\begin{equation}
\begin{split}
    u(k-1)&=K x(k-1)\\
    u(k-1)&=\begin{bmatrix}k_1& k_2& \cdots& k_n\end{bmatrix} \begin{bmatrix}x_1(k-1)\\x_2(k-1)\\\vdots\\x_n(k-1)\end{bmatrix}\\
    u(k-1)&=k_1x_1(k-1)+k_2x_2(k-1)+\cdots+k_nx_n(k-1)
\end{split}
\end{equation}
olarak yazılabilir. Bu kontrolör ile kapalı çevrim durum uzay modeli
\begin{equation}
    \begin{split}
        x(k)&=A x(k-1)+Bu(k-1),\quad y(k-1)=C x(k-1)\\
        x(k)&=A x(k-1)+BKx(k-1),\quad y(k-1)=C x(k-1)\\
        x(k)&=(A+BK) x(k-1),\quad y(k-1)=C x(k-1)
    \end{split}
\end{equation}
olarak elde edilir. Kapalı çevrim modelin z tanım bölgesi ifadesi
\begin{equation}
    \begin{split}
        x(k)&=(A+BK) x(k-1)+B r(k-1),\quad y(k-1)=C x(k-1)\\
        z^1 x(k-1)&=(A+BK) x(k-1)+B r(k-1),\quad y(k-1)=C x(k-1)\\
        (zI-(A+BK)) x(k-1)&=B r(k-1),\quad y(k-1)=C x(k-1)\\
        x(k-1)&=(zI-(A+BK))^{-1} B r(k-1),\quad y(k-1)=C x(k-1)\\
        y(k-1)&=C(zI-(A+BK))^{-1}B r(k-1)\\
        \frac{y(k-1)}{r(k-1)}&=C(zI-(A+BK))^{-1}B
    \end{split}
\end{equation}
şeklindedir ve karakteristik polinom
\begin{equation}
    \begin{split}
        p_c(z)=det(zI-(A+BK))
    \end{split}
\end{equation}
ile hesaplanır. Bu polinom için kutuplar seçilirken $K$ kontrolör matrisi hesaplanır. Bu işlem için
\begin{equation}
    p_d(z)=-\begin{bmatrix}0& 0& \cdots& 0& 1\end{bmatrix}\begin{bmatrix}B& AB& \cdots& A^{n-1}B\end{bmatrix}^{-1}p_d(A)
\end{equation}
burada $p_d(z)$ atanmak istenen polinom olmak üzere formülü kullanılabilir.   

