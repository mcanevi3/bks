\chapter{Z ve S tanım bölgesi}
Zaman tanım bölgesinden S tanım gölgesine dönüşüm
\begin{equation}
\begin{split}
    F(s)&=\mathcal{L}\{f(t)\}=\sum_{k=0}^{\infty}f(kT)e^{-kTs}\\
    &=f(0)+f(T)e^{-Ts}+f(2T)e^{-2Ts}+\cdots\\
\end{split}
\end{equation}
olarak verilmiştir. Zaman tanım bölgesinden Z tanım bölgesine geçiş ise
\begin{equation}
    \begin{split}
        F(z)&=\mathcal{Z}\{f(t)\}=\sum_{k=0}^{\infty}f(kT)z^{-k}\\
        &=f(0)+f(T)z^{-1}+f(2T)z^{-2}+\cdots
    \end{split}
\end{equation}
şeklindedir. S ve Z tanım bölgesi dönüşümlerine dikkat edilirse
\begin{equation}
    \begin{split}
        \mathcal{L}\{f(t)\}&=\sum_{k=0}^{\infty}f(kT)(e^{Ts})^{-k}\\
        \mathcal{Z}\{f(t)\}&=\sum_{k=0}^{\infty}f(kT)z^{-k}\\
\end{split}
\end{equation}
ifadelerinden
\begin{equation}
    z=e^{sT}
\end{equation}
ilişkisi elde edilir.

Z dönüşümü için tablo Tablo~\ref{tbl:ztransform1} ile verilmiştir.
\setlength{\tabcolsep}{40pt}
\begin{table}[!htb]
    \centering
    \caption{S ve Z dönüşümü tablosu}
    \label{tbl:ztransform1}
    \begin{tabularx}{\textwidth}{ccc}\hline
        Zaman domeni& $F(s)$& $F(z)$\\[5pt]\hline
        $\delta(t)$& $1$& $1$\\[5pt]
        $\delta(t-kT)$& $e^{-kTs}$& $z^{-k}$\\[5pt]
        $u(t)=1$& $\frac{1}{s}$& $\frac{z}{z-1}$\\[5pt]
        $t$& $\frac{1}{s^2}$& $\frac{Tz}{(z-1)^2}$\\[5pt]
        $e^{-at}$& $\frac{1}{s+a}$& $\frac{z}{z-e^{-aT}}$\\[5pt]
        $1-e^{-at}$& $\frac{a}{s(s+a)}$& $\frac{z(1-e^{-aT})}{(z-1)(z-e^{-aT})}$\\[5pt]
        $sin(wt)$& $\frac{w}{s^2+w^2}$& $\frac{zsin(wT)}{(z-1)(z^2-2zcos(wT)+1)}$\\[5pt]
        $cos(wt)$& $\frac{s}{s^2+w^2}$& $\frac{z(z-cos(wT))}{(z-1)(z^2-2zcos(wT)+1)}$\\[5pt]\hline
    \end{tabularx}
\end{table}