\newcommand\UniversiteAdi{Niğde Ömer Halisdemir Üniversitesi}
\newcommand\BolumAdi{MEKATRONİK BÖLÜMÜ}
\newcommand\DersKodu{MKT2002}
\newcommand\DersAdi{BİLGİSAYARLI KONTROL SİSTEMLERİ}
\newcommand\SinavAdi{Ödev 1}
\newcommand\SinavTarihi{10.03.2025}
\newcommand\SinavSaati{10:00}
\newcommand\SinavSuresi{90dk}

\pagestyle{fancy}
\fancyhf{} % clear existing header/footer entries
\fancyhead[R]{Öğrenci No:\hspace{4.5cm}}
\fancyhead[L]{Ad Soyad:\hspace{7cm}}
\noindent
\begin{tabular}{
    p{0.15\linewidth}
    p{0.15\linewidth}
    p{0.3\linewidth}
    p{0.1\linewidth}
    p{0.15\linewidth}}
    \multicolumn{5}{c}{\textbf{\BolumAdi}}\\
    \multicolumn{5}{c}{\textbf{\DersAdi}}\\\hline
    \multicolumn{1}{|r|}{Ders Kodu:}&
    \multicolumn{1}{|c|}{\DersKodu}&
    \multicolumn{1}{|c|}{}& 
    \multicolumn{1}{|r|}{Tarih:}&
    \multicolumn{1}{|c|}{\SinavTarihi} \\\hline
    \multicolumn{1}{|r|}{Sınav Türü:}&
    \multicolumn{1}{|c|}{\SinavAdi}&  
    \multicolumn{1}{|c|}{}&
    \multicolumn{1}{|r|}{Saat:}&
    \multicolumn{1}{|c|}{\SinavSaati}\\\hline
    \multicolumn{1}{|r|}{Dönemi:}&
    \multicolumn{1}{|c|}{2024-2025}&
    \multicolumn{1}{|c|}{}&
    \multicolumn{1}{|r|}{Süre:}&
    \multicolumn{1}{|c|}{\SinavSuresi} \\\hline
    &&&&\\
\end{tabular}\\\\
\noindent\begin{center}
\begin{tabular}{|r|c|c|c|c|c|}\hline
    \textbf{Soru:}&
    \textbf{1a}&
    \textbf{1b}&
    \textbf{2a}&
    \textbf{2b}&
    \textbf{Toplam}\\\hline
    \textbf{Puan:}&
    \textbf{25}&
    \textbf{25}&
    \textbf{25}&
    \textbf{25}&
    \textbf{100}\\\hline
    \textbf{Not:}&&&&&\\\hline
\end{tabular}\end{center}
\noindent\textbf{Uyarı:}
\begin{itemize}\bfseries
    \item Soruları dikkatlice okuyunuz. Hesap makinesi kullanılabilir.
    \item İşlemleri atlamadan ve ayrıntılı olarak veriniz. Sadece nümerik yanıtlar veya çizimler ara işlemler olmadan kabul edilmemektedir.
    \item Yuvarlamalar 2 hane yapılacaktır. $\mathbf{1.99456\approx1.99}$ olarak alınacaktır.
\end{itemize}
\noindent\textbf{Soru:} Aktif süspansiyon sistemi için diferansiyel denklem 
\begin{equation}
\begin{split}
    m_1\frac{d^2x_1}{dt^2}&=-b_1\left(\frac{dx_1}{dt}-\frac{dx_2}{dt}\right)-k_1(x_1-x_2)+u\\
    m_2\frac{d^2x_2}{dt^2}&=b_2\left(\frac{dx_1}{dt}-\frac{dx_2}{dt}\right)+b_2\left(\frac{dw}{dt}-\frac{dx_2}{dt}\right)+k_2(w-x_2)-u
\end{split}
\end{equation}
olarak verilmiştir. Fark denklemlerini elde ediniz.
\textbf{Extra:}Fark denklemlerini kullanarak $u$ girişine sıfır ve $w$ girişine birim basamak uygulayınız ve $x_1$ ve $x_2$ değişkenlerini çiziniz.
